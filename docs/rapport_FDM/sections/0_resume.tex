\section*{Résumé}
\addcontentsline{toc}{section}{Résumé}

Ce projet traite de la détection de fraudes par chèque dans un contexte de données fortement déséquilibrées, où moins de 1\% des transactions sont frauduleuses. Nous exploitons un jeu de données réel issu de la grande distribution française, comprenant 4,6 millions de transactions sur 10 mois (février à novembre 2017) décrites par 23 variables. Face à ce déséquilibre extrême (taux de fraude de 0,60\% en apprentissage et 0,88\% en test), nous implémentons et comparons plusieurs approches : techniques de rééchantillonnage (SMOTE, sous-échantillonnage), algorithmes de classification supervisés (régression logistique, Random Forest, XGBoost), et méthodes \textit{cost-sensitive} adaptées aux contraintes métier.

\textbf{Partie 1 (Optimisation de la F-mesure) :} Neuf approches différentes ont été testées, incluant des combinaisons de rééchantillonnage (SMOTE, UnderSampling), de pondération des classes (\textit{class\_weight}, \textit{scale\_pos\_weight}) et d'ajustement du seuil de décision. Le meilleur modèle identifié est \textbf{XGBoost avec seuil ajusté ($\tau=0.939$)}, atteignant une F-mesure de \textbf{0.148} sur l'ensemble de test (précision : 18.4\%, rappel : 12.4\%). Ce résultat représente une amélioration significative par rapport aux baselines (F1 $\approx$ 0.03-0.09) et surpasse les performances rapportées dans la littérature sur des datasets similaires (F1 $\approx$ 0.08). L'ajustement fin du seuil de décision s'avère être la stratégie la plus efficace pour maximiser la F-mesure, bien que le rappel modeste (12.4\% des fraudes détectées) souligne la difficulté intrinsèque de la tâche face à un déséquilibre de 1:165.

\textbf{Partie 2 (Optimisation du profit) :} Les mêmes modèles ont été réévalués selon un critère de profit intégrant une matrice de coûts réaliste (gains de 5\% sur transactions acceptées, pertes variables de 0-80\% sur fraudes acceptées). Le meilleur modèle pour le profit est \textbf{XGBoost avec seuil ajusté ($\tau=0.85$)}, générant une marge de \textbf{2 055 747€} sur 3 mois (~685k€/mois). Ce modèle diffère du meilleur modèle pour la F-mesure par son seuil de décision plus bas, privilégiant un rappel plus élevé (25.7\% vs. 12.4\%) au détriment de la précision (6.5\% vs. 18.4\%). Cette différence se traduit par un gain de marge de \textbf{+45 891€} (+2,3\%) par rapport au modèle optimisé pour la F-mesure, démontrant que la F-mesure ne constitue \textit{pas} un proxy fiable du profit réel.

\textbf{Conclusion principale :} L'optimisation directe du critère métier (profit) est essentielle pour maximiser la rentabilité. Les résultats révèlent une faible corrélation entre F-mesure et marge, justifiant l'importance d'aligner les métriques d'évaluation avec les objectifs business. Le système proposé pourrait générer environ 8,2M€ de marge annuelle, avec des perspectives d'amélioration via optimisation des hyperparamètres, réentraînement adaptatif et monitoring continu.

\vspace{0.5cm}
\noindent\textbf{Mots-clés :} Détection de fraudes, apprentissage déséquilibré, rééchantillonnage, cost-sensitive learning, F-mesure, optimisation de profit, XGBoost.

\newpage
